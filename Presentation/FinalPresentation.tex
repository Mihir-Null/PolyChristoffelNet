\documentclass[aspectratio=169]{beamer}

% -------------------------
% Theme / packages
% -------------------------
\usetheme{metropolis}
\usepackage{amsmath,amssymb,mathtools}
\usepackage{bm}
\usepackage{graphicx}
\usepackage{booktabs}

\title{PolyChristoffel Networks}
\subtitle{Obtaining latent manifolds via curvature learning}
\author{Mihir Talati}
\date{\today}

% -------------------------
% Helpful macros
% -------------------------
\newcommand{\R}{\mathbb{R}}
\newcommand{\dd}{\,\mathrm{d}}
\newcommand{\Gammaijk}{\Gamma^i_{\;jk}}
\newcommand{\vx}{\dot{x}}
\newcommand{\vxx}{\ddot{x}}

\begin{document}

\maketitle

% =========================================================
% SECTION 1: MOTIVATION + BASIC BACKGROUND / INTUITION
% =========================================================
\section{Motivation \& Intuition}

\begin{frame}{Why another dynamics model?}
\begin{itemize}
  \item Most current NNs are considered Black-Box simulators.
  \item We know something useful hides in Latent Space (see AlexNet) but we can't extract it due to arbitrary nonlinearities.
  \item Modern Models solve this by imposing structure on embeddings and evolutions.
  \item Physics models such as HNNs/LNNs do this via:
  \begin{itemize}
    \item invariances, conservation laws, interpretable coefficients
    \item extrapolation beyond the training distribution
  \end{itemize}
  \item However we suspect that the geometry of latent spaces encode these properties
\end{itemize}
\end{frame}

\begin{frame}{Geometric intuition: ``forces'' as coordinate effects and nonlinearity}
\begin{itemize}
  \item A straight-line path in one coordinate system can look curved in another.
  \item Example: inertial motion in Cartesian becomes nontrivial in polar coordinates.
  \item The apparent ``forces'' come from geometry via \textbf{Christoffel symbols}.
\end{itemize}
\vspace{0.5em}
\[
\vxx^i + \Gamma^i_{\;jk}(x)\,\vx^j \vx^k = 0
\]
\begin{itemize}
  \item Interpret $\Gamma^i_{\;jk}(x)$ as \textbf{state-dependent couplings} of velocities into acceleration.
  \item Goal here: make the \textbf{equations of motion} a first-class object:
  \[
    \text{learn } \Gamma(x) \text{ so that trajectories follow a geodesic equation.}
  \]
\end{itemize}
\end{frame}

\begin{frame}{Graduate DiffGeom in 30 seconds: metric, connection, geodesics}
\begin{itemize}
  \item A (pseudo-)Riemannian metric $g(x)$ defines local inner products:
  \[
  \langle u, v\rangle_x = u^\top g(x)\,v
  \]
  \item The Levi--Civita connection (torsion-free, metric-compatible) yields Christoffels:
  \[
  \Gamma^i_{\;jk}(g)=\tfrac12 g^{i\ell}
  \Big(\partial_j g_{k\ell}+\partial_k g_{j\ell}-\partial_\ell g_{jk}\Big)
  \]
  \item Geodesics are ``straightest'' paths under that connection:
  \[
  \vxx^i + \Gamma^i_{\;jk}(x)\,\vx^j \vx^k = 0.
  \]
\end{itemize}
\end{frame}

% =========================================================
% SECTION 2: MODEL EXPLANATION (majority)
% =========================================================
\section{Model}

\begin{frame}{Model overview}
\textbf{Inputs:} initial state/latent $x_0$, initial velocity $v_0$ (and optionally observed trajectory).\\
\textbf{Learn:} a polynomial Christoffel field $\Gamma(x)$ (and optionally an autoencoder).

\vspace{0.75em}
\begin{center}
\begin{tabular}{@{}c@{}c@{}c@{}}
Observed $y_t$ & \hspace{0.5em} & Latent/state $x_t$ \\
\hline
$x_t=E(y_t)$ & $\Rightarrow$ & $\Gamma(x)$ produces geodesic rollout $\hat{x}_{0:T}$ \\
$\hat{y}_t=D(\hat{x}_t)$ & & Losses enforce dynamics + metric consistency
\end{tabular}
\end{center}

\vspace{0.5em}
\textbf{Design principle:} keep the dynamics layer interpretable by making $\Gamma(x)$ a low-degree polynomial.
\end{frame}

\begin{frame}{Parameterization: polynomial Christoffel network}
Let $x\in\R^d$. Build monomial features $\phi(x)\in\R^P$ (degree $\le p$):
\[
\phi(x)=\big[1,\; x_1,\dots,x_d,\; x_1^2,\; x_1x_2,\dots\big].
\]
Then define
\[
\Gamma^i_{\;jk}(x) = \sum_{m=1}^{P} C^i_{\;jk,m}\,\phi_m(x).
\]
\begin{itemize}
  \item $C^i_{\;jk,m}$ are trainable coefficients (directly interpretable).
  \item Enforce torsion-free symmetry by construction:
  \[
  \Gamma^i_{\;jk} = \Gamma^i_{\;kj}.
  \]
\end{itemize}
\end{frame}

\begin{frame}{Dynamics block: geodesic equation as a neural layer}
Use first-order state $z=(x,v)$ with $v=\dot{x}$:
\[
\dot{x}=v,\qquad \dot{v}^i = -\Gamma^i_{\;jk}(x)\,v^j v^k.
\]
This defines a deterministic vector field $f_\theta(z)$ where parameters are the polynomial coefficients.

\vspace{0.5em}
\textbf{Interpretability:}
\begin{itemize}
  \item acceleration is quadratic in velocity and structured by $\Gamma(x)$
  \item no arbitrary activation needed in the dynamics law
\end{itemize}
\end{frame}

\begin{frame}{Integrator: Scuffed ResNet rollout (discrete geodesic flow)}
A simple differentiable integrator (semi-implicit Euler):
\[
v_{t+1}=v_t + \Delta t\,a(x_t,v_t),\quad
x_{t+1}=x_t + \Delta t\,v_{t+1},
\]
where
\[
a^i(x,v) = -\Gamma^i_{\;jk}(x)\,v^j v^k.
\]

\begin{itemize}
  \item Equivalent to stacking residual blocks with shared parameters.
  \item Stable, CUDA-friendly, easy to backprop through long rollouts.
  \item Later swap-in: RK4 / differentiable ODE solvers if needed.
\end{itemize}
\end{frame}

\begin{frame}{Optional: jointly learned autoencoder with manifold constraint}
When observations live in $y$-space (images, fields, etc.), use an encoder/decoder:
\[
x_t = E_\psi(y_t),\qquad \hat{y}_t = D_\psi(\hat{x}_t).
\]
To enforce latent validity on a chosen manifold $\mathcal{M}$:
\[
x \leftarrow \Pi_{\mathcal{M}}(x)
\]
(e.g., sphere projection, hyperboloid retraction).\\

\begin{itemize}
  \item Joint training: geometry losses backprop through $E_\psi$.
  \item Encourages a chart where dynamics are ``as geodesic as possible.''
\end{itemize}
\end{frame}

\begin{frame}{Why metric reconstruction / pseudo-Riemannian regularization?}
Learning $\Gamma(x)$ freely can fit trajectories but yield a connection that is not Levi--Civita of any metric.

\vspace{0.5em}
To encourage a \textbf{(pseudo-)Riemannian interpretation}, impose constraints consistent with:
\begin{itemize}
  \item torsion-free: $\Gamma^i_{\;jk}=\Gamma^i_{\;kj}$
  \item metric compatibility: $\nabla g = 0$
  \item fixed signature $(p,q)$, non-degeneracy ($\det g \neq 0$)
\end{itemize}
\end{frame}

\begin{frame}{Metric reconstruction idea: integrate the compatibility equation}
Metric compatibility in coordinates:
\[
\partial_i g_{jk} = \Gamma^\ell_{\;ij}\,g_{\ell k} + \Gamma^\ell_{\;ik}\,g_{j\ell}.
\]
Treat this as a (path) ODE along a curve $x(s)$ from a basepoint $x_0$:
\[
\frac{\dd}{\dd s}g_{jk}(s)
=
\Big(\Gamma^\ell_{\;ij}(x(s))g_{\ell k}(s) + \Gamma^\ell_{\;ik}(x(s))g_{j\ell}(s)\Big)\frac{\dd x^i}{\dd s}.
\]

\begin{itemize}
  \item Choose $g(x_0)=g_0$ with fixed signature; integrate to get $g(x)$.
  \item If $\Gamma$ is not metric-realizable, the reconstructed $g(x)$ becomes \textbf{path-dependent}.
\end{itemize}
\end{frame}

\begin{frame}{Consistency loss: penalize path dependence (loop loss)}
Compute $g(x)$ via two different paths:
\begin{itemize}
  \item Path A: straight line $x_0 \to x$
  \item Path B: two-segment $x_0 \to x_m \to x$
\end{itemize}
Then define:
\[
\mathcal{L}_{\text{loop}} = \|g^{(A)}(x) - g^{(B)}(x)\|_F^2.
\]
\textbf{Interpretation:}
\begin{itemize}
  \item pushes $\Gamma(x)$ toward connections that preserve some bilinear form
  \item provides extra signal even if you only supervise trajectories
\end{itemize}
\end{frame}

\begin{frame}{Pseudo-Riemannian validity losses (practical)}
Given reconstructed $g(x)$:
\begin{itemize}
  \item Symmetry:
  \[
  \mathcal{L}_{\text{sym}}=\|g-g^\top\|_F^2
  \]
  \item Non-degeneracy barrier (avoid singular metric):
  \[
  \mathcal{L}_{\det} = \text{softplus}\big(\alpha - \log|\det g|\big)
  \]
  \item Fixed signature $(p,q)$ via eigenvalue sign penalties:
  \[
  \mathcal{L}_{\text{sig}}=\sum_{i=1}^d \text{softplus}\big(-\beta\, s_i\lambda_i(g)\big),
  \quad s_i\in\{+1,-1\}.
  \]
\end{itemize}
\end{frame}

\begin{frame}{Overall objective and training loop}
\textbf{Primary fit:} match trajectories (latent or decoded):
\[
\mathcal{L}_{\text{traj}}=\frac{1}{T}\sum_{t}\|\hat{x}_t-x_t\|^2
\quad \text{or}\quad
\mathcal{L}_{\text{recon}}=\frac{1}{T}\sum_{t}\|D(\hat{x}_t)-y_t\|^2.
\]
\textbf{Regularize geometry:}
\[
\mathcal{L}
=
\mathcal{L}_{\text{traj/recon}}
+\lambda_{\text{loop}}\mathcal{L}_{\text{loop}}
+\lambda_{\det}\mathcal{L}_{\det}
+\lambda_{\text{sig}}\mathcal{L}_{\text{sig}}
+\lambda \|C\|_2^2.
\]

\begin{itemize}
  \item Backprop through (i) rollout integrator and (ii) metric reconstruction.
  \item Coefficients $C$ remain interpretable (polynomial terms in $\Gamma$).
\end{itemize}
\end{frame}

\begin{frame}{Training: backprop through geodesic rollout (computational graph)}
\textbf{Forward pass (one minibatch)}
\begin{itemize}
  \item Initialize from data (or latent): $(x_0, v_0)$.
  \item Rollout dynamics for $t=0,\dots,T-1$ with step $\Delta t$:
  \[
  v_{t+1}=v_t + \Delta t\,a(x_t,v_t),\qquad
  x_{t+1}=x_t + \Delta t\,v_{t+1},
  \]
  where
  \[
  a^i(x_t,v_t) = -\Gamma^i_{\;jk}(x_t)\,v_t^j v_t^k,
  \qquad
  \Gamma^i_{\;jk}(x)=\sum_{m=1}^P C^i_{\;jk,m}\,\phi_m(x).
  \]
  \item Compute loss on the trajectory (and optional geometry regularizers):
  \[
  \mathcal{L}=\sum_{t=1}^T \|x_t - x_t^{\text{true}}\|^2 + \lambda\,\mathcal{L}_{\text{geom}}.
  \]
\end{itemize}

\vspace{0.5em}
\textbf{Autodiff view:} the rollout is a differentiable computation graph. Gradients flow
\[
\mathcal{L}\ \to\ (x_{1:T},v_{1:T})\ \to\ a(\cdot)\ \to\ \Gamma(x_t)\ \to\ C.
\]
\end{frame}

\begin{frame}{Gradient signal on Christoffel coefficients (why it is interpretable)}
Because $\Gamma$ is \textbf{linear} in the coefficients $C^i_{\;jk,m}$, the gradient has a clean form.

\vspace{0.25em}
\textbf{At a single step $t$:}
\[
\Gamma^i_{\;jk}(x_t)=\sum_{m=1}^P C^i_{\;jk,m}\,\phi_m(x_t)
\quad\Longrightarrow\quad
\frac{\partial \Gamma^i_{\;jk}(x_t)}{\partial C^p_{\;qr,m}}
=
\delta^i_p\,\delta^q_j\,\delta^r_k\;\phi_m(x_t).
\]

\textbf{Acceleration coupling:}
\[
a^i_t = -\Gamma^i_{\;jk}(x_t)\,v_t^j v_t^k
\quad\Longrightarrow\quad
\frac{\partial a^i_t}{\partial C^i_{\;jk,m}}
=
-\phi_m(x_t)\,v_t^j v_t^k.
\]

\textbf{Chain rule through time (BPTT):}
\[
\frac{\partial \mathcal{L}}{\partial C^i_{\;jk,m}}
=
\sum_{t=0}^{T-1}
\left\langle
\frac{\partial \mathcal{L}}{\partial a^i_t},
-\phi_m(x_t)\,v_t^j v_t^k
\right\rangle
\;+\;\frac{\partial \mathcal{L}_{\text{geom}}}{\partial C^i_{\;jk,m}}.
\]

\vspace{0.25em}
\textbf{Interpretation:} each coefficient update is a weighted sum of \emph{state features} $\phi_m(x_t)$ times \emph{velocity interactions} $v_t^j v_t^k$,
so learned geometry can be inspected term-by-term (e.g., which monomials drive which couplings).
\end{frame}

\begin{frame}{Proof of Concept experiment: polar inertial motion}
Use a synthetic benchmark with known Christoffels:
\[
\Gamma^r_{\theta\theta}=-r,\qquad \Gamma^\theta_{r\theta}=\Gamma^\theta_{\theta r}= \frac{1}{r}
\]
(others $=0$), derived from Euclidean plane in polar coordinates.

\begin{itemize}
  \item Train $\Gamma(x)$ to reproduce trajectories.
  \item Use diagnostics to compare learned components to ground truth.
  \item Verify torsion-free symmetry and reduced path dependence in reconstructed $g$.
\end{itemize}
\end{frame}

\begin{frame}{Diagnostics I: Learned Christoffels vs. ground truth}
  \textbf{What we plot}\\[0.25em]
  Evaluate the learned connection on a grid $(r,\theta)$ and extract:
  \[
    \Gamma^r_{\theta\theta}(r,\theta),\quad \Gamma^\theta_{r\theta}(r,\theta)
  \]
  Compare to ground truth:
  \[
    \Gamma^r_{\theta\theta}=-r,\qquad
    \Gamma^\theta_{r\theta}=\frac{1}{r}.
  \]
  \vspace{0.25em}
  \textbf{How to interpret}
  \begin{itemize}
    \item \textbf{$\theta$-invariance:} GT depends only on $r$, so learned heatmaps should be \emph{nearly constant across $\theta$}.
    \item \textbf{Shape:} $\Gamma^r_{\theta\theta}$ should look like a linear ramp in $r$; $\Gamma^\theta_{r\theta}$ should decay like $1/r$.
    \item \textbf{Torsion-free check:} $\Gamma^\theta_{r\theta}\approx \Gamma^\theta_{\theta r}$.
  \end{itemize}
\end{frame}

\begin{frame}{Christoffel Diagnostics}
   \begin{figure}
    \centering
    \includegraphics[width=0.7\linewidth]{figs/AE0PolarFig1.png}
    \caption{\scriptsize Learned vs. GT Christoffel components on $(r,\theta)$ grid.}
    \includegraphics[width=0.7\linewidth]{figs/AE1PolarChrist.png}
    \caption{\scriptsize Learned vs. GT Christoffel components on \textbf{Embedded} $(r,\theta)$ grid.}
  \end{figure}
\end{frame}

\begin{frame}{Diagnostics II: Reconstructed metric and path-dependence (loop error)}
\begin{columns}[T,onlytextwidth]
  \column{1.0\textwidth}
  \textbf{Metric target (polar plane)}\\[0.25em]
  The Euclidean metric in polar coordinates is:
  \[
    g(r,\theta)=
    \begin{pmatrix}
      1 & 0 \\
      0 & r^2
    \end{pmatrix}.
  \]
  We reconstruct $g$ from the learned $\Gamma$ by integrating metric-compatibility:
  \[
    \nabla g = 0 \quad\Rightarrow\quad \partial_i g_{jk} = \Gamma^\ell_{ij} g_{\ell k} + \Gamma^\ell_{ik} g_{j\ell}.
  \]
  \vspace{0.25em}
  \textbf{How to interpret}
  \begin{itemize}
    \item \textbf{Diagonal structure:} $g_{r\theta}\approx 0$ (off-diagonal heatmap near zero).
    \item \textbf{Component shapes:} $g_{rr}\approx 1$ (flat), $g_{\theta\theta}\propto r^2$ (quadratic in $r$).
    \item \textbf{Path dependence:} reconstruct $g$ via two paths; the loop error
    \[
      \|g^{(A)}(x) - g^{(B)}(x)\|_F^2
    \]
    should be small if $\Gamma$ is close to Levi--Civita of some metric.
  \end{itemize}
\end{columns}
\end{frame}

\begin{frame}{Pretty Graphs Pt 2}
  Reconstructed $g$ components and loop error heatmap.
    \begin{figure}
    \centering
    \includegraphics[width=0.7\linewidth]{figs/metric.png}
    \caption{\scriptsize Learned Metric}
  \end{figure}
\end{frame}

\begin{frame}{Pretty Graphs Pt 2 Pt 2}
  \begin{figure}
    \centering
    \includegraphics[width=0.7\linewidth]{figs/truemetric.png}
    \caption{\scriptsize True Metric}
  \end{figure}
\end{frame}

\begin{frame}{AutoEncoded metrics)}
  While the AutoEncoded metrics are a little harder to interpret due to nonlinear coordinate transforms.
  \begin{figure}
    \centering
    \includegraphics[width=0.7\linewidth]{figs/ae_metric.png}
    \caption{\scriptsize Learned Metric}
  \end{figure}
\end{frame}

\begin{frame}{AutoEncoded metrics)}
  We can still see what metric values we get by reinterpreting the embeddings under the ground truth metric.
  \begin{figure}
    \centering
    \includegraphics[width=0.7\linewidth]{figs/kindatrue_ae_metric.png}
    \caption{\scriptsize "True"*** Metric}
  \end{figure}
\end{frame}

% =========================================================
% SECTION 3: FUTURE DIRECTIONS
% =========================================================
\section{Future Directions}

\begin{frame}{Scaling up: leverage strong autoencoders (CNNs, etc.)}
\begin{itemize}
  \item Many domains already have excellent representation learners:
  \begin{itemize}
    \item CNN autoencoders / VAEs for images and fields
    \item pretrained encoders with semantic latent factors
  \end{itemize}
  \item Proposal: \textbf{freeze or lightly finetune} an existing encoder/decoder,
  then learn $\Gamma(x)$ (and optionally $g(x)$) on the latent manifold to obtain:
  \begin{itemize}
    \item interpretable latent dynamics
    \item geometry-aware interpolation/extrapolation (geodesics)
    \item constraints like signature / nondegeneracy for stability
  \end{itemize}
  \item The notion of enforcing a Pseudo-Riemannian metric alsso defines a consistent, invariant dot product for vectors at any point on the manifold, which is also precisely how many encoders are trained (e.g in NLP)
\end{itemize}
\end{frame}

\begin{frame}{Research directions}
\begin{itemize}
    \item \textbf{Metric Intervals:} Enforce a notion of "distance" via states through known equivalencies
    \item \textbf{Lorentzian Geometry:} Allow reparametrization of evolution along axes other than time
  \item \textbf{Beyond geodesics:} add forcing/dissipation in a structured way
  (e.g., geodesic $+$ learned potential or Rayleigh dissipation term).
  \item \textbf{Better metrizability:} directly co-train $g(x)$ and enforce
  $\Gamma_{\text{poly}}\approx \Gamma(g)$ to guarantee Levi--Civita structure.
  \item \textbf{Topology/coverage:} multiple charts or implicit atlas, while retaining interpretability.
  \item \textbf{Stochastic extensions:} geodesic drift with learned diffusion (SDE in latent).
\end{itemize}
\end{frame}

\begin{frame}{Takeaways}
\begin{itemize}
  \item Christoffel symbols provide an interpretable parameterization of dynamics:
  \[
  \vxx^i = -\Gamma^i_{\;jk}(x)\,\vx^j\vx^k
  \]
  \item Polynomial $\Gamma(x)$ yields a small, inspectable model with calculus-friendly smoothness.
  \item Metric reconstruction and pseudo-Riemannian regularizers provide extra training signals and geometric validity checks.
  \item Natural next step: apply on strong latent spaces (e.g., CNN-based autoencoders) to learn \textbf{curved latent dynamics}.
\end{itemize}
\end{frame}

\begin{frame}[allowframebreaks]{References}
\small
\begin{thebibliography}{9}
\bibitem{chen2018node}
T. Q. Chen, Y. Rubanova, J. Bettencourt, and D. Duvenaud,
``Neural Ordinary Differential Equations,'' arXiv:1806.07366, 2018.

\bibitem{greydanus2019hnn}
S. Greydanus, M. Dzamba, and J. Yosinski,
``Hamiltonian Neural Networks,'' arXiv:1906.01563, 2019.

\bibitem{korman2021atlas}
E. O. Korman,
``Atlas Based Representation and Metric Learning on Manifolds,'' arXiv:2106.07062, 2021.

\bibitem{burge2025ngf}
J. B\"urge,
``Neural geodesic flows,'' Master’s thesis, ETH Z\"urich, 2025.

\bibitem{he2025lorentz}
N. He, M. Yang, and R. Ying,
``Lorentzian Residual Neural Networks,'' arXiv:2412.14695, 2025.
\end{thebibliography}
\end{frame}


\end{document}
